\chapter{Wstęp}
\section{Wprowadzenie}
Temat niniejszej pracy nie jest przypadkowy. Zastosowanie sztucznej inteligencji z dnia na dzień staje się coraz popularniejsze. Wielkie firmy wykorzystują ją do szybkiego rozpoznawania między innymi obrazów, dźwięków lub tekstów.

Inspiracją do napisania takiego systemu jest ciągła konieczność poprawiania algorytmów, szukających takich samych lub podobnych produktów ze sklepów internetowych w aktualnie wykonywanej pracy zawodowej. Problemem jest tu zbyt duża różnorodność dotycząca kategorii danych produktów. Towar z jednej kategorii ma w opisie inne informacje niż produkt z drugiej kategorii. System do klasteryzacji dokumentów tekstowych pozwoli na częściowe rozwiązanie tego problemu. Dzięki temu można w łatwy sposób wygenerować listę dokumentów tekstowych, które będą składać się z opisów, nazw, kategorii czy specyfikacji produktów. Następnie należałoby uruchomić algorytm, który pogrupuje dokumenty tekstowe w takie grupy, w których obiekty będą do siebie jak najbardziej podobne.

System ten będzie bardzo przydatny w branży E-Commerce, której pracownicy będą szybciej analizować ceny na rynku z różnych sklepów internetowych czy kategorii danych produktów. System sprawdzi się również do grupowania innych rodzajów dokumentów tekstowych. W poniższej pracy wykorzystam go do pogrupowania artykułów powiązanych ze sobą tematycznie, co przedstawię w kolejnych rozdziałach.


\section{Cel i zakres pracy}
Celem pracy jest utworzenie systemu, który na wejściu otrzymuję listę dokumentów tekstowych, a na wyjściu generuje grupy tych dokumentów dokonując selekcji poszczególnych dokumentów do grup na podstawie treści dokumentów. Oznacza to, że odległość pomiędzy dokumentem tekstowym z jednej grupy, a dokumentem z innej musi być jak największa, natomiast dla obiektów znajdujących się w tej samej grupie - jak najmniejsza.

Praca będzie swym zakresem obejmowała:
\begin{itemize}
    \item dokonanie przeglądu bibliotek do wstępnego przetwarzania i klasteryzacji dokumentów tekstowych,
    \item przygotowanie testowego zbioru dokumentów tekstowych,
    \item czyszczenie, normalizacja oraz transformacja dokumentów tekstowych,
    \item zastosowanie algorytmów klasteryzacji do budowy systemu klasteryzacji treści dokumentów tekstowych,
    \item wykonanie testów weryfikujących skuteczność opracowanego systemu.
\end{itemize}

\section{Struktura pracy}

Praca została podzielona na dwie części, teoretyczną oraz praktyczną. Część teoretyczna została zaprezentowana w rozdziale 2. W ramach tej części opisano metody przetwarzania dokumentów tekstowych, klasteryzacji oraz sposobów weryfikacji wyników. W rozdziale 3 zaprezentowano przykładowe wyniki dla metod, które opisano w rozdziale 2. Wyniki przedstawione są w formie rysunków oraz tabel wraz z komentarzem.