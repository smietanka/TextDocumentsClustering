\chapter{Podsumowanie}

W pracy zostały poruszone tematy związane z analizą tekstu oraz grupowaniem dokumentów tekstowych, wykorzystując w tym celu metody czyszczenia, normalizacji oraz reprezentacji tekstu za pomocą miary $TF-IDF$. Na dzień dzisiejszy, analizowanie dokumentów tekstowych jest dość trudnym zagadnieniem.  Oprócz samej budowy zdań w tekście, należy również rozumieć sens słów, ponieważ w niektórych sytuacjach to właśnie sens słowa jest istotniejszym elementem, decydującym o przynależności dokumentu do konkretnej grupy tekstów. 

Praca ta prezentuje poszczególne kroki podjęte dla osiągnięcia celu. Omówione zostały pojęcia teoretyczne konkretnych technologii, instalacja oraz konfiguracja środowiska pracy Visual Studio Code oraz języka programowania Python. Opisane zostały możliwości wykorzystywanych bibliotek do wstępnego przetwarzania tekstu.

W kolejnych częściach pracy przedstawione zostały etapy przygotowań dokumentów tekstowych do wykorzystania podczas grupowania tekstu, oraz algorytm, który został wykorzystany do osiągnięcia zamierzonego celu.

Aby dokonać weryfikacji dokładności klasteryzacji dokumentów tekstowych opracowano metodę dopasowywania klastrów wygenerowanych przez algorytm k-means do etykiet referencyjnych. Do tego celu wykorzystane zostało przedstawienie danych na histogramach, oraz zamiana klas wyjściowych z wykorzystaniem różnych mapowań.

Przedstawiony w niniejszej pracy system jest możliwy do uruchomienia na innych zbiorach dokumentów tekstowych. Przykładowe zbiory takich dokumentów dostępne są w załączniku na płycie CD dołączonej do pracy.
