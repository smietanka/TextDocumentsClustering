\subsection*{Streszczenie}

Niniejsza praca przedstawia realizację projektu, którego celem jest opracowanie systemu, który w sposób nienadzorowany będzie w stanie skategoryzować dokumenty tekstowe na podstawie ich treści.

Na początku pracy przedstawiona została teoria dotycząca eksploracji danych. Opisana została ścieżka instalacji oraz konfiguracji środowiska pracy i wykorzystanego języka programowania. Omówiono możliwości wykorzystanych bibliotek do wstępnego przetwarzania i klasteryzacji dokumentów tekstowych oraz wykorzystanych w dalszej części metod i technik do ich obróbki. 

W kolejnej części znajdują się szczegółowe informacje dotyczące wybranych technologii do uzyskania zamierzonego celu. Wytłumaczone zostały poszczególne etapy przygotowania dokumentów tekstowych, m.in. czyszczenie, normalizacja oraz transformacja dokumentów tekstowych. 

Ostatni etap pracy poświęcono na badania eksperymentalne, których celem było zweryfikowanie skuteczności przygotowanego systemu. Badania przeprowadzono na zbiorze rzeczywistych dokumentów tekstowych pochodzących z ponad 2000 źródeł wiadomości.

\vspace{1cm}
\noindent\textbf{Słowa kluczowe:} sztuczna inteligencja, dokumenty tekstowe, klasteryzacja